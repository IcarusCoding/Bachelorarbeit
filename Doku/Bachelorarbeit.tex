\documentclass[a4paper,titlepage,bibtotoc,liststotoc,final,twoside,openright,cleardoubleempty,BCOR12mm]{scrreprt} % ABGABEFASSUNG

\usepackage[utf8]{inputenc}
\usepackage[T1]{fontenc}
\usepackage[ngerman]{babel}
\usepackage{epsfig}
\usepackage{amsmath}
\usepackage{hyperref}
\usepackage{bibgerm}
\usepackage{breakurl}
\usepackage{ae,aecompl}
\usepackage{graphicx}
\usepackage{tabularx}
\usepackage{rotating}
\usepackage{lscape}
\usepackage{subfigure}
\usepackage{xcolor}
\usepackage{xargs}
\usepackage[colorinlistoftodos,prependcaption,textsize=tiny]{todonotes}
\usepackage{acronym}

% Abstände bei Trennzeichen für Zahlen korrekt machen
\mathcode`,="013B
\mathcode`.="613A

% Tex ist kein Zeichenprogramm...
\usepackage{tikz}
\usetikzlibrary{arrows,automata,shapes,topaths}

% Vorlage - Obere Linie, Seitenzahlen
\usepackage{scrlayer-scrpage}
\pagestyle{scrheadings}
\clearscrheadfoot
\automark[section]{chapter}
\setheadsepline{.4pt}
\ihead[]{\leftmark}
\ohead[]{\rightmark}
\ofoot[\pagemark]{\pagemark}

% Index erstellen
\usepackage{makeidx}
\makeindex

% Arbeitsbeschreibung & Makros
\newcommand{\Name}{Deniz Groenhoff}
\newcommand{\Email}{deniz.groenhoff@uni-oldenburg.de}
\newcommand{\MatrNr}{5477417}
\newcommand{\Titel}{Annotationsbasierte Einstiegserleichterung in die Entwicklung von JavaFX-Anwendungen}
\newcommand{\Department}{Department für Informatik}
\newcommand{\Abteilung}{Medieninformatik und Multimedia-Systeme}
\newcommand{\ErstGutachter}{Prof. Dr. Susanne Boll}
\newcommand{\ZweitGutachter}{Dr.-Ing. Dietrich Boles}
\newcommand{\ArbeitTyp}{Bachelorarbeit}

\newcommandx{\add}[2][1=]{\todo[linecolor=red,backgroundcolor=red!25,bordercolor=red,#1]{#2}}

\newcommand{\MONTH}{%
	\ifcase\the\month
	\or Januar% 1
	\or Februar% 2
	\or März% 3
	\or April% 4
	\or Mai% 5
	\or Juni% 6
	\or Juli% 7
	\or August% 8
	\or September% 9
	\or Oktober% 10
	\or November% 11
	\or Dezember% 12
	\fi}

\title{\Titel}
\author{\Name}
\date{\today}

\begin{document}

% Fuers inhaltsvz gibts andere seitenzahlen
\pagenumbering{Roman}

% Titelseite erstellen
\begin{titlepage}
	\thispagestyle{empty}
% zentriert: 0.26cm
\hoffset=0.4cm
\begin{center}
\begin{figure}
\centering
\includegraphics[width=10cm]{Abbildungen/uniol_4c}\\
\end{figure}
\vspace{2cm}
{\bf \Department}\\Abteilung für \Abteilung\\
\vspace{4cm}
{\bf \ArbeitTyp}\\
\vspace{.5cm}
{\begin{minipage}[t]{10cm}
\centering
\Large
\Titel
\end{minipage}}\\
\vspace{3cm}
\Name\\
\vspace{.5cm}
\today\\
\vspace{3cm}
1. Gutachterin: \ErstGutachter\\
2. Gutachter: \ZweitGutachter
\end{center}

\end{titlepage}

\newpage
\thispagestyle{empty}
\quad

% Eigenstaendigkeitserklaerung
\newpage
\thispagestyle{empty}
\vspace{6.8cm}
\begin{center}
	\begin{Large}
		\textbf{Erklärung}
	\end{Large}
\end{center}

\noindent
Ich erkläre an Eides statt, dass ich die Arbeit selbständig verfasst
und keine anderen als die angegebenen Quellen und Hilfsmittel benutzt und die allgemeinen Prinzipien wissenschaftlicher Arbeit und Veröffentlichungen, wie sie in den Leitlinien guter wissenschaftlicher Praxis der Carl von Ossietzky Universität Oldenburg festgelegt sind, befolgt habe.

\vspace{3cm}
\noindent
\Name\\
Matrikelnummer \MatrNr\\
Oldenburg, den \the\day. \MONTH\ \the\year{}


% Zusammenfassung erstellen
\begin{abstract}
	\section*{Zusammenfassung}
Hier kommt in der Regel eine ca. halbseitige Zusammenfassung von Motivation und
Ergebnis der Arbeit hin. Eine zeitliche Abfolge, wann was gemacht wurde, spielt hier
keine Rolle
\footnote{Fussnote 1}
	\begin{center}
	\Large\bfseries{Abstract}
\end{center}
Abstract (jeweils auf Englisch und Deutsch)
\end{abstract}
% Inhaltsverzeichnis erstellen
\tableofcontents

\pagenumbering{arabic} % Seitenzahlen umstellen

\chapter{Einleitung}
\label{einleitung}
Im Rahmen der Digitalisierung ist das Programmieren zugänglicher und massentauglicher als je zuvor und es wird immer mehr auf Fertigkeiten wie das effiziente und weitgehend fehlerfreie Programmieren, vor allem in sicherheitskritischen Umgebungen, gesetzt. Die Verwendung von Quelltextmetadaten kann verschiedenen Quelltextelementen einen hohen Grad an Struktur geben. Diese Metadaten sind in vielen Programmiersprachen aber auch außerhalb der Informatik als Annotationen\footnote{Wenn in der Arbeit über Annotationen gesprochen wird, ist immer von Java-Annotationen auszugehen (außer anders angegeben)} bekannt. Diese sind ein dynamisches Programmierkonstrukt, welches immer populärer wird und in den meisten Fällen für eine Vereinfachung von komplexen Aufgaben, der Automation von repetitiven Vorgängen und der Dokumentation von Quelltextelementen sorgen kann.
\section{Motivation}
\label{motivation}
Zur Erstellung von platformunabhängigen Java-Anwendungen mit grafischer Benutzungsschnittstelle wird immer mehr auf JavaFX, den Nachfolger der Swing Bibliothek, gesetzt. Dies erlaubt es Entwicklern, mit relativ wenig Zeitaufwand zeitgemäße und moderne Applikationen zu erstellen. Selbst Entwickler, welche über keinerlei Erfahrung mit der Entwicklung von grafischen Oberflächen in der Java Programmiersprache verfügen, werden keine Schwierigkeiten haben, solche mittels JavaFX zu entwerfen und schließlich zu implementieren. Wie später in der Problemanalyse gezeigt, bietet JavaFX zwar einen einfachen Einstieg in grundlegende Programmierkonstrukte, kann aber bei der Erstellung von komplexen Anwendungen dennoch schnell zeitaufwendig und fehleranfällig werden.
\section{Zielsetzung}
\label{zielsetzung}
In dieser Arbeit soll aufgrund der simplifizierenden Natur von Annotationen in anderen Teilgebieten der Programmierung eine Bibliothek entwickelt werden, welche verschiedene Aspekte im Entwicklungsprozess von JavaFX Anwendungen durch die Verwendung von Annotationen vereinfacht oder vollständig ersetzt. Dazu soll der Funktionsumfang erweitert werden, indem bereits vorhandene Funktionen dynamischer gestaltet oder vollständig neue Funktionen implementiert werden. Konzepte aus anderen Bibliotheken für die grafische Benutzeroberflächenentwicklung sollen dabei ebenfalls in Betracht gezogen werden. Auch wenn bei der Implementierung hauptsächlich auf die Nutzung von Annotationen zurückgegriffen wird, sollen ebenfalls unabhängige Schnittstellen entwickelt werden, welche vom Entwickler genutzt werden können.
\section{Struktur}
\label{struktur}
Die Struktur der Arbeit setzt sich aus sechs Teilen zusammen. Begonnen wird mit der Definition und Erklärung von fundamentalen Kernkonzepten, aus welchen ein optimales Verständnis der genutzten Technologien resultieren soll. Zu diesen Kernkonzepten gehören essentielle softwaretechnische Grundlagen wie Entwurfsmuster oder das Vorstellen und die anschließende nähere Beleuchtung der Annotationsfunktion in Java und anderen Programmiersprachen. Außerdem wird auf einzelne Komponenten von JavaFX eingegangen und deren Interaktionen untereinander mit Beispielen untermauert (\autoref{grundlagen}).\\
Danach werden aktuelle Beispiele zur Annotationsprogrammierung aus anderen Arbeiten, Bibliotheken und Programmiersprachen vorgestellt. Dabei liegt der primäre Fokus auf einer Vereinfachung des Entwicklungsprozesses in der Java Umgebung. Der Stand der Technik wird abschließend in einem Fazit im zweiten Teil der Arbeit zusammengefasst und es erfolgt ein kurzer Ausblick auf Funktionalitäten in JavaFX, welche in bisherigen Arbeiten/Bibliotheken nicht zu finden waren (\autoref{stand_der_technik}).\\
Nachdem die gegenwärtig verfügbaren Maßnahmen zur Simplifizierung durch die Verwendung von Annotationen vorgestellt wurden, wird mit dem nächsten Teil, dem Entwickeln eines Entwurfes, begonnen. Dazu erfolgt zu Beginn eine Problemanalyse, welche Schwierigkeiten und komplexe Strukturen bei der Programmierung von JavaFX Anwendungen identifiziert. Die gefundenen Probleme werden dann mithilfe einer Anforderungsanalyse durch funktionale und nicht funktionale Anforderungen dargestellt. Dazu werden notwendige Rahmenbedingungen an die Bibliothek und die Softwarearchitektur erhoben und verwendete externe Bibliotheken kurz vorgestellt. Außerdem werden Konzepte für essentielle Schnittstellen entworfen, die einerseits für eine Gewährleistung der Funktionalität notwendig sind und anderseits für den Bibliotheksnutzer unabhängig verwendbar gemacht werden, sowie alle zu implementierenden Annotationen mit ihren Funktionen und Limitationen vorgestellt (\autoref{konzeption_und_entwurf}).\\
Mit den aus dem Entwurf und dem Konzept resultierenden Problemlösungen, Rahmenbedingungen an die Architektur und Anforderungen, wird das System in \fullref{implementierung} prototypisch implementiert. Dazu werden die implementierten Subsysteme näher betrachtet und teilweise mit Diagrammen untermauert. Darüber hinaus werden wichtige Teile des Entwicklungsprozesses aufgegriffen und in Form von Quelltextausschnitten erklärt.\\
Nachdem der Entwicklungsprozess des Systems beendet wurde, wird eine Anwendung zur Demonstration der Funktionen erstellt, welche durch das Nutzen von den hinzugefügten Annotationen als Hilfestellung für eventuelle Nutzer der Bibliothek dienen soll. Dabei wird explizit das Zusammenspiel der einzelnen Subsysteme vorgestellt. Darauf folgt ein Vergleich der implementierten Funktionen mit denen von JavaFX, mit einem besonderen Fokus auf Aspekte wie die Benutzerfreundlichkeit oder dem Programmieraufwand (\autoref{evaluation}).\\
Im letzten Kapitel erfolgt eine Zusammenfassung der erarbeiteten Ergebnisse sowie eine Bewertung dieser. Des Weiteren wird auf nicht funktionale Anforderungen wie die Erweiterbarkeit und die Wartbarkeit des Systems eingegangen (\autoref{fazit}).
\chapter{Grundlagen}
\label{grundlagen}
\add{Correction}
\noindent In diesem Kapitel werden die theoretischen Grundlagen von essentiellen Komponenten dieser Arbeit erläutert. Dazu wird die Relevanz von Entwurfsmustern justifiziert und auf zwei bedeutende Muster näher eingegangen. Diese sind sowohl erforderlich für die folgenden Kapitel als auch für das Verständnis der softwaretechnischen Prinzipien von JavaFX.\\
Danach wird die JavaFX-Bibliothek vorgestellt und fundamentale Konzepte wie beispielsweise die auf der \ac{xml} basierende Layouting-Sprache erläutert.\\
Abschließend wird das generelle Annotationenkonzept in der Informatik mit speziellen Fokus auf die Programmiersprache Java erklärt. Dabei werden die verschiedenen Annotationstypen näher beschrieben und jeweils mit Beispielen untermauert, sowie die Möglichkeiten der eigentlichen Auswertung von Annotationen skizziert.

\section{Entwurfsmuster}
\label{entwurfsmuster}

\add{Intro}

\subsection{Definition}
\label{entwurfsmuster_definition}

\add{Definition Entwurfsmuster}

\subsection{Notwendigkeit}
\label{entwurfsmuster_notwendigkeit}

\add{Notwendigkeit \& Justifikation von Entwurfsmustern}
\newpage
\section{JavaFX}
\label{javafx}
\add{glossar für fxml z.B.?}
\noindent JavaFX ist eine auf Java basierte, quelloffene Bibliothek für das Entwickeln von grafischen Benutzerschnittstellen für Client Applikationen. Im Vergleich zum Vorgänger GUI-Toolkit Java-Swing, bietet JavaFX ein modernes, zeitgemäßes Design der allgemeinen Benutzeroberfläche sowie den dort enthaltenen Schaltflächen und Komponenten \cite{Sharan2015}. Kombiniert mit den objektorientierten Konzepten von Java, ist JavaFX in der Lage auch komplexe nebenläufige Anwendungen mit vielen Abhängigkeiten darzustellen und aufgrund der Plattformunabhängigkeit auch ohne viele Restriktionen in allen bekannten Betriebssystemen einsetzbar.\\
Dazu ist JavaFX auch weitgehend konform mit bekannten Entwurfsmustern der Softwareentwicklung wie beispielsweise dem \ac{mvc}- oder dem Beobachter-Muster, weshalb implementierte Anwendung selbst bei vielen \ac{loc}, eine grundsätzlich hohe Strukturiertheit auf Quelltextebene aufweisen. Das grafische Layout kann dabei nicht ausschließlich durch Java-Quelltext sondern auch mittels der an die \ac{xml} angelehnte Markup-Sprache FXML erstellt werden. Letzteres kann durch externe Tools wie dem Scene-Builder enorm vereinfacht werden \cite{Vos2018}.

\subsection{Aufbau und Szenengraph}
\label{javafx_szenengraph}
Damit eine JavaFX-Anwendung als solche identifiziert werden kann, muss die Hauptklasse von der \texttt{Application}-Klasse erben. Die Namensgebung der Klassen, welche für die Struktur bzw. den Aufbau einer JavaFX-Anwendung zuständig sind, basiert auf Begriffe der Theaterumgebung \cite{Anderson2019}:
\begin{description}
	\item Die \textbf{\texttt{Stage}} Klasse repräsentiert ein Anwendungsfenster, welches das Design des Fensterlayouts des aktuell genutzten Betriebssystems nutzt. Eine \texttt{Stage} ist teilweise modifizierbar, so können beispielsweise die Standardschaltflächen in der Titelleiste entfernt order deaktiviert werden. Werden mehrere Fenster benötigt, so können nach dem Initialisieren der Haupt-\texttt{Stage} durch die JavaFX-Plattform, manuell Weitere hinzugefügt werden.
	\item Die \textbf{\texttt{Scene}} Klasse ist für das Layout und die Darstellung von vorhandenen oder selbsterstellten JavaFX-Komponenten verantwortlich. Jede Komponente, welche durch eine \texttt{Scene}-Instanz angezeigt und verwaltet werden soll, wird in einer hierarchisch angeordneten, objektorientierten Datenstruktur eingefügt, welche in der Computergrafik als Szenengraph bekannt ist \cite{Hughes2013}. Jeder \texttt{Stage} muss zwangsläufig eine \texttt{Scene} zugewiesen werden.
	\item Die \textbf{\texttt{Node}} Klasse ist eine darstellbare Komponente im Szenengraphen wie beispielsweise eine Schaltfläche oder ein Containerelement. \texttt{Node} Instanzen im Szenengraph können Kindelemente enthalten und maximal einem Elternelement zugeordnet sein. Der Szenengraph ähnelt somit einer Baumstruktur mit einer Wurzel und einem oder mehreren Blättern. Damit eine \texttt{Node}-Instanz Kindelemente besitzen darf, muss diese immer von der abstrakten \texttt{Parent}-Klasse erben. Das Layouting und die Positionierung im lokalen Koordinatensystem wird bei vorhandenen Kindelementen immer durch das Elternelement kontrolliert. Jede darzustellende Komponente muss von der \texttt{Node}-Klasse erben \cite{Juneau2013}.
\end{description}
Ein minimales Beispiel für eine voll funktionsfähige JavaFX-Anwendung, welche das Zusammenspiel der oben genannten Konzepte und Klassen widerspiegelt, ist in \autoref{lst:example_javafxapp} dargestellt.

\begin{figure}[H]
	\begin{lstlisting}[caption={Beispiel -- Minimale JavaFX-Anwendung.}, captionpos=b, label=lst:example_javafxapp]
		public class TestApplication extends Application {
		
			public static void main(String[] args) {
				launch(args);
			}
			
			@Override
			public void start(Stage primaryStage) {
				final Pane root = new Pane();
				root.getChildren().add(new Button("TestButton"));
				final Scene scene = new Scene(root, 250, 250);
				primaryStage.setScene(scene);
				primaryStage.show();
			}
		
		}
	\end{lstlisting}
\end{figure}

\subsection{Properties und Bindings}

\subsection{Layouting: FXML vs. Quelltext}
Wie in der Einleitung schon angedeutet, ist es möglich das Layout der Anwendung auch per FXML zu erstellen. Eine Prävention von Boilerplate-Code kann durch das Auslagern von häufig verwendeten JavaFX-Komponenten in externe FXML-Dateien erfolgen \cite{Kruk2018}. Das Verwenden von solchen Dateien sorgt für eine bessere Trennung von Controllern und Logik im Sinne des z.B. \ac{mvc}-Entwurfsmusters \cite{Juneau2013} und durch die hohe Konfigurierbarkeit sind für eine eventuelle Veröffentlichung der Applikation wichtige Konzepte wie die Internationalisierung, leichter umzusetzen \cite{Steyer2014}. Durch das Parsen und Aufbauen des Szenengraphen zur Laufzeit des Programms ist eine Verwendung von FXML-Dateien jedoch langsamer als benötigte Komponenten direkt im Java Quelltext zu deklarieren. Fast alle JavaFX-Nodes können ohne Weiteres in XML-Elementen verwendet und angepasst werden. Außerdem ist es möglich, direkt eine manuell erstellte Controller-Klasse mit einer FXML-Datei zu assoziieren. Das Laden einer FXML-Datei und das darauffolgende Aufbauen des Szenengraphen wird durch die \texttt{FXMLLoader}-Klasse durchgeführt. Das Beispiel aus \autoref{lst:example_javafxapp} ist als FXML-Datei in \autoref{lst:example_fxmllayouting} zu erkennen.

\begin{figure}[H]
	\begin{lstlisting}[caption={Beispiel -- FXML Layouting.}, captionpos=b, label=lst:example_fxmllayouting, language=XML]
	<?xml version="1.0" encoding="UTF-8"?>
	
	<?import javafx.scene.layout.Pane?>
	<?import javafx.scene.control.Button?>
	
	<Pane xmlns="http://javafx.com/javafx">
		<Button>TestButton</Button>
	</Pane>
	\end{lstlisting}
\end{figure}
\add{example with controller}
\subsection{Scene-Builder}

\section{Java-Annotationen}
\label{java_annotationen}
\noindent Annotationen sind in der Sprachwissenschaft eine Möglichkeit einen vorhandenen Text mit Anmerkungen zu versehen für beispielsweise Disambiguierung, also das Eliminieren von Mehrdeutichkeiten eines Wortes oder für das Erklären von komplexen Textabschnitten. Sie geben dem Leser Zusatzinformationen um Sachverhalte einfacher darzustellen und sorgen dadurch für ein schnelleres bzw. besseres Verständnis des Textes. Dabei sind solche Anmerkungen kein Hauptbestandteil von Texten sondern dienen ausschließlich als Ergänzung.\\
In der Informatik sind Annotationen ebenfalls nur ein deskriptives Strukturkonzept, welche es dem Entwickler ermöglicht, verschiedenen strukturellen Elementen der Programmierung (wie Felder oder Klassen), Metadaten zuzuweisen \cite{Yu2019}. Das Nutzen von Annotationen in Anwendungen ist aufgrund ihrer meist simpel gehaltenen Syntax auch für Programmiereinsteiger vorteilhaft und durch ihre Anpassungsfähigkeit und Flexibilität sind sie in vielen Bibliotheken und Programmiersprachen vertreten.
\subsection{Definition}
\label{java_annotationen_definition}
\add{Reference}
\add{Move footnote to first occurence}
\noindent Annotationen \footnote{Wenn in der Arbeit über Annotationen gesprochen wird, ist immer von Java-Annotationen auszugehen (außer anders angegeben)} wurden mit Java 5 (2014) in die Sprache eingeführt und werden seitdem immer häufiger für verschiedene Aspekte der Programmierung genutzt \cite{Rocha2011}. Mit ihnen kann eine Steuerung des Compilers erfolgen, eine Verarbeitung der Metadaten zu Kompilierzeit durchgeführt werden oder das Verhalten von Anwendungen zu Laufzeit modifiziert oder gelenkt werden \cite{Yu2019}. Aufgrund der Tatsache, dass es sich nur um rein deskriptive Metadaten handelt, ist es Annotationen nicht direkt möglich mit existierendem Quelltext zu interagieren.  Möglichkeiten zur Verarbeitung dieser Metadaten werden in Sektion \ref{java_annotation_laufzeitauswertung} vorgestellt. Neben den von Java vordefinierten Annotationen wie z.B. \texttt{@Override} für das Überschreiben von vererbten Methoden oder \texttt{@SuppressWarnings} für das Unterdrücken von Compilerwarnungen, können auch eigene Annotationen deklariert werden.\\
Es handelt sich bei Annotationen in Java um spezialisierte Schnittstellen bei welchen das \texttt{interface}-Schlüsselwort durch ein \texttt{@}-Zeichen Präfix zu \texttt{@interface} erweitert wird \cite{Gosling2005}. Außerdem ist es Annotationen nicht erlaubt wie bei normalen Schnittstellendefinitionen das Schlüsselwort \texttt{extends} für eine Vererbung zu verwenden, da die Superschnittstelle implizit vom Compiler auf die \texttt{Annotation} Klasse des \texttt{java.lang.annotation} Pakets gesetzt wird \cite{Oracle2017}. Ein Beispiel einer  Annotationsdefinition ist in \autoref{lst:annotation_definition} dargestellt.
\begin{figure}[H]
	\centering
	\begin{lstlisting}[caption={Beispiel einer Annotationsdefinition.}, captionpos=b, label=lst:annotation_definition]
	public @interface TestAnnotation {
	    // ...
	}
	\end{lstlisting}
\end{figure}
\noindent In der Analogie des Kapitels \ref{java_annotationen} können Elemente mit strukturgebenden Charakter wie Bestandteile eines Satzes annotiert werden. Analog dazu sind in der Java-Programmierung Klassen, Methoden, Felder etc. für die Strukturierung des Quelltextes und der Softwarearchitektur verantwortlich und somit auch mit Annotationen erweiterbar. Um Sprachelemente zu annotieren muss wie in \autoref{lst:annotated_example} dargestellt, ein \texttt{@}-Präfix zum eigentlichen Klassennamen hinzugefügt werden.
\begin{figure}[H]
	\centering
	\begin{lstlisting}[caption={Beispiel einer annotierten Klasse.}, captionpos=b, label=lst:annotated_example]
	#@TestAnnotation
	public class TestClass {
	    // ...
	}
	\end{lstlisting}
\end{figure}
\noindent Aufgrund der besonders einfachen Syntax und dem vergleichsweise geringen Aufwand, ist ein steigender Trend der Nutzung von Java-Annotationen in Open-Source Anwendungen zu erkennen. Werden Annotationen jedoch übermäßig verwendet, so kann es schnell zu Quelltext-Verschmutzung kommen, was im Kontext der Annotationsprogrammierung auch \glqq annotation hell\grqq{} (dt. Annotationshölle) genannt wird. Annotationen erreichen dann das Gegenteil des gewünschten Zwecks -- Statt den Entwicklungsprozess vereinfachend zu unterstützen, wird der Quelltext schwer nachvollziehbar und wirkt unstrukturiert und unübersichtlich.\\
Dennoch zeigt eine Studie aus dem Jahre 2011, welche 1094 quelloffene GitHub-Projekte auf die Verwendung von Annotationen untersucht hat, dass javabasierte Anwendungen und Bibliotheken, bei aktiver Nutzung von Annotationen, eine geringere Fehleranfälligkeit aufweisen \cite{Rocha2011}.
\subsection{Syntax}
\label{java_annotationen_anwendung}
\add{lst design}
\noindent Annotationen können Attribute besitzen, welche bei Kompilierzeit bzw. Laufzeit ausgelesen werden können. Die Typen dieser Attribute sind nicht vollständig frei wählbar -- So ist es beispielsweise nicht möglich ein Attribut vom Typen \texttt{Object} in einer Annotation zu kapseln, ohne einen Kompilierfehler auszulösen. Erlaubt sind alle primitiven bzw. atomaren Datentypen und Instanzen der \texttt{String}-, \texttt{Class}- und \texttt{Enum}-Klasse sowie eindimensionale Arrays aus den vorherigen Typen. Außerdem ist es möglich, Attributen einen voreingestellten Wert mittels des Schlüsselwortes \texttt{default} zuzuweisen \cite{Gosling2005}. Annotationen müssen in einer der folgenden Syntaxen benutzt werden:
\begin{description}
	\item \textbf{Normal Annotations} sind ganz normal deklarierte Annotationen, bei welchen die Attribute mittels Aufzählung in Klammern übergeben werden.
	\begin{figure}[H]
		\noindent
		\newlength\heightone
		\begin{adjustbox}{minipage=[t]{.45\linewidth},gstore totalheight=\heightone,margin=\fboxsep+\fboxrule}
			\begin{lstlisting}[caption=Deklaration -- Normal Annotation., captionpos=b, label=lst:decl_normal]
public @interface Entity {
	String name();
	int id();
}
			\end{lstlisting}
		\end{adjustbox}\hfill
		\begin{adjustbox}{minipage=[t][\heightone]{0.5\linewidth}}
			\begin{lstlisting}[caption=Anwendung -- Normal Annotation, captionpos=b, label=lst:appl_normal]
#@Entity(name="test", id=2)
public class TestEntity {
	// ...
}
			\end{lstlisting}
		\end{adjustbox}
	\end{figure}
	\item \textbf{Single-Element Annotations} sind eine Kurzform der normalen Annotationen mit einem \texttt{value}-Attribut und keinen weiteren nicht-default Attributen.
	\begin{figure}[H]
		\noindent
		\begin{adjustbox}{minipage=[t]{.45\linewidth},gstore totalheight=\heightone,margin=\fboxsep+\fboxrule}
			\begin{lstlisting}[caption=Deklaration -- Single-Element Annotation., captionpos=b, label=lst:decl_single]
public @interface Entity {
	String value();
	int id() default -1;
}
			\end{lstlisting}
		\end{adjustbox}\hfill
		\begin{adjustbox}{minipage=[t][\heightone]{0.5\linewidth}}
			\begin{lstlisting}[caption=Anwendung -- Single-Element Annotation, captionpos=b, label=lst:appl_single]
#@Entity("test")
public class TestEntity {
	// ...
}
			\end{lstlisting}
		\end{adjustbox}
	\end{figure}
	\item \textbf{Marker Annotations} sind ebenfalls eine Kurzform der normalen Annotationen mit keinen oder nur default Attributen.
	\begin{figure}[H]
		\noindent
		\begin{adjustbox}{minipage=[t]{.45\linewidth},gstore totalheight=\heightone,margin=\fboxsep+\fboxrule}
			\begin{lstlisting}[caption=Deklaration -- Marker Annotation., captionpos=b, label=lst:decl_marker]
public @interface Entity {
	String name() default "";
	int id() default -1;
}
			\end{lstlisting}
		\end{adjustbox}\hfill
		\begin{adjustbox}{minipage=[t][\heightone]{0.5\linewidth}}
			\begin{lstlisting}[caption=Anwendung -- Marker Annotation, captionpos=b, label=lst:appl_marker]
#@Entity
public class TestEntity {
	// ...
}
			\end{lstlisting}
		\end{adjustbox}
	\end{figure}
\end{description}
\noindent Die Sichtbarkeit von eigenen Annotationen zu verschiedenen Phasen des Codezyklus kann durch die von Java bereitgestellte Annotation \texttt{@Retention} gesteuert werden. Das übergebene Enum-Attribut klassifiziert die Annotation dann in einen von drei Typen \cite{Rocha2011}:
\begin{description}
	\item \textbf{Quellcode-Annotationen} sind nur beim Kompiliervorgang auslesbar und können dem Compiler Anweisungen geben oder mithilfe von Annotation-Prozessoren z.B. neue Klassen automatisch generieren. Sie sind in der kompilierten Java-Anwendung nicht mehr erhalten.
	\item \textbf{Klassen-Annotationen} sind nach dem Kompilierungsprozess noch in der Anwendung erhalten und können durch externe Tools wie z.B. dem Code-Obfuskator ProGuard ausgelesen werden.
	\item \textbf{Laufzeit-Annotationen} sind nach der Kompilierung und beim Start der Anwendung erhalten und können dann mithilfe der Reflection-API zur Laufzeit ausgewertet werden.
\end{description}
\noindent Des Weiteren kann gesteuert werden, welche Typen der Strukturelemente eines Quellcodes annotiert werden können. Ein Beispiel für eine zur Laufzeit beibehaltene Annotation, welche nur an Methoden angebracht werden kann ist in \autoref{lst:full_annotation_example} zu erkennen.
\begin{figure}[H]
	\centering
	\begin{lstlisting}[caption={Beispiel einer Laufzeit Annotation.}, captionpos=b, label=lst:full_annotation_example]
	
	#@Target(ElementType.(@\tikzmark{aLeft}{}@)METHOD(@\tikzmark{aRight}{}@))
	#@Retention(RetentionPolicy.(@\tikzmark{bLeft}{}@)RUNTIME(@\tikzmark{bRight}{}@))
	public @interface Event {
		int id();
		int priority() default 0;
	}
	(@
	\begin{tikzpicture}[overlay,remember picture]
		\foreach \x/\y in {a/red, b/blue} {
			\DrawOverBar[-, \y, thick]{\x Left.north}{\x Right.north}
		}
		\node[draw](onlymethods) at (8.5,3) {Nur an Methoden};
		\node[draw](runtime) at (8.5,1.5) {Zur Laufzeit};
		\DrawArrow[red, in=-180]{a}{onlymethods}{-4.8em, 0}
		\DrawArrow[blue, in=-270, out=20]{b}{runtime}{0, 0.65em}
	\end{tikzpicture}
	@)
	\end{lstlisting}
\end{figure}

\add{Add compile time annotation processing if used in this thesis}
\subsection{Auswertung von Laufzeit-Annotationen}
\label{java_annotation_laufzeitauswertung}
Für eine Auswertung von Laufzeit-Annotationen, muss zwangsläufig die Reflection-API von Java genutzt werden. Wenn eine Programmiersprache eine Form von Reflection (dt. Spiegelung) aufweist, so ist es möglich Attribute, Logikfluss und andere Eigenschaften während der Laufzeit zu ändern. In objektorientierten Sprachen wie Java wird diese \glqq computational reflection\grqq{} genutzt, um die Möglichkeit einer Selbstbeobachtung der eigenen Sprachelemente zu schaffen \cite{Li2017}. Die API ermöglicht somit beispielsweise das Auslesen von Laufzeit-Annotationen und deren deklarierte Attribute oder das dynamische Instanziieren von Klassen \cite{Forman2004}. Jedes Java-Element der Reflection API (Feld, Methode, Klasse, ...), welches annotierbar ist, wird durch die Vererbung der \texttt{AnnotatedElement}-Klasse als solches klassifiziert \cite{Schildt2019}. Damit nun alle vorhandenen Annotation ausgelesen werden können, kann die Methode \inlinecode{java}{AnnotatedElement#getDeclaredAnnotations} aufgerufen werden \cite{Pigula2015}. Das Lesen der Attribute der in \autoref{lst:full_annotation_example} vordefinierten Annotation ist in \autoref{lst:annotation_processing_example} zu erkennen.
\begin{figure}[H]
	\begin{lstlisting}[caption={Auslesen einer Laufzeit-Annotation.}, captionpos=b, label=lst:annotation_processing_example]
    if(Test.class.isAnnotationPresent(Event.class)) {
	    Event e = Test.class.getDeclaredAnnotation(Event.class);
	    int id = e.id();
	    int priority = e.priority();
    }
	\end{lstlisting}
\end{figure}

\subsection{Beispiele der Annotationsprogrammierung}
\label{java_annotationen_annotationsprogrammierung}

\add{Beispiele der Annotationsprogrammierung}
\chapter{Stand der Technik}
\label{stand_der_technik}

\add{Intro}

\section{Aktuelle Verwendung von Annotationen}
\label{aktuelle_verwendung_von_annotationen}

\add{Intro}

\subsection{JavaFX}
\label{aktuelle_verwendung_von_annotationen_javafx}

\add{JavaFX Beispiele}

\subsection{Android}
\label{aktuelle_verwendung_von_annotationen_android}

\add{Android Beispiele}

\subsection{JavaX}
\label{aktuelle_verwendung_von_annotationen_javax}

\add{JavaX Beispiele (z.B. JAXB)}

\section{Maßnahmen zur Simplifizierung des Entwicklungsprozesses}
\label{maßnahmen_zur_simplifizierung_des_entwicklungsprozesses}

\add{Intro}

\subsection{Workflow Optimierung}
\label{maßnahmen_zur_simplifizierung_des_entwicklungsprozesses_workflow}

\add{Workflow Optimierung}

\subsection{Vereinfachung durch gesteigerte Übersichtlichkeit}
\label{maßnahmen_zur_simplifizierung_des_entwicklungsprozesses_übersichtichkeit}

\add{Vereinfachung durch gesteigerte Übersichtlichkeit}

\subsection{Fazit}
\label{maßnahmen_zur_simplifizierung_des_entwicklungsprozesses_fazit}

\add{Fazit}
\chapter{Konzeption und Entwurf}
\label{konzeption_und_entwurf}
In diesem Kapitel werden mögliche Probleme bei der Entwicklung sowie bei der Nutzung von JavaFX Anwendungen identifiziert. Dabei wird ein besonderer Fokus auf das Finden von Architekturmängeln, fehlenden Funktionalitäten und verbesserungswürdigen Techniken gelegt. Um eine Fehleranfälligkeit zu reduzieren, sollen komplexe und sich häufig wiederholende Quelltextbausteine automatisch erstellt oder durch Annotationen vereinfacht werden. Die vollständige Substitution eines aufwendige Prozesses ist dabei ebenfalls möglich. Probleme, Vereinfachungen oder Verbesserungen sollen durch das Untersuchen von vorhandenen, quelloffenen JavaFX-Projekten und Bibliotheken gefunden werden. Auch sollen Ideen und Konzepte zusammengetragen werden, welche auf JavaFX anwendbar sind, jedoch nur in anderen Bibliotheken und Frameworks aufzufinden sind. \\
Bei der Problemanalyse wird stets das Ziel verfolgt, das Entwickeln mit JavaFX zu vereinfachen -- besonders für noch unerfahrene Entwickler. Danach wird eine Anforderungsanalyse durchgeführt, mit welcher systematisch funktionale sowie nichtfunktionale Anforderungen auf der Basis der gefundenen Probleme erstellt werden. \\
Auf die Anforderungserhebung folgt die Konzeption des benötigten Systems und der zugrundeliegenden Architektur. Essentielle Komponenten werden mit \ac{uml} Diagrammen entworfen und im Detail erläutert. Bei der Existenz verschiedener Lösungsstrategien für ein Problem, wird jede Strategie einzeln beleuchtet und nach Kriterien wie Sinnhaftigkeit und Machbarkeit entschieden, welche für das System am besten geeignet ist. Wichtige Richtlinien wie die angestrebte Softwarequalität werden ebenfalls beschrieben.

\section{Identifikation von Problemen und komplexen Strukturen in der JavaFX Entwicklung}
\label{problemanalyse}
\add{08.06}

\section{Anforderungsanalyse}
\label{anforderungsanalyse}
In der Anforderungsanalyse werden die gefundenen Problemlösungen und Vereinfachungen aus \autoref{problemanalyse} in Form von funktionalen und nichtfunktionalen Anforderungen formuliert. Dabei werden die Anforderungen in zwei Klassen unterteilt:
\begin{description}
	\item \textbf{Fundamentale Anforderungen} sind Anforderungen, welche für eine Funktion des Systems essentiell sind, alle genannten Probleme weitgehend beheben und daher zwangsläufig implementiert werden müssen.
	\item \textbf{Optionale Anforderungen} sind Anforderungen, welche keinen Einfluss auf eine ordnungsgemäße Funktionalität des Systems haben. Sie sind optional und werden möglicherweise aufgrund ihrer Komplexität nur teilweise oder gar nicht implementiert und können stattdessen für eine Erweiterung des Systems durch weitere Entwickler genutzt werden.
\end{description}
\add{format of requirements}

\subsection{Funktionale Anforderungen}
\label{anforderungsanalyse_funktional}
Im Folgenden werden alle funktionalen Anforderungen definiert. Sie beschreiben alle gewünschten Funktionen des Endproduktes.
\add{09.06}

\add{Funktionale Anforderungen als Unterpunkte}
\subsubsection{...}

\subsection{Nichtfunktionale Anforderungen}
\label{anforderungsanalyse_nichtfunktional}
Im Folgenden werden alle nichtfunktionalen Anforderungen definiert. Sie beschreiben Qualitätseigenschaften an das System wie Möglichkeiten der Erweiterbarkeit und Wartbarkeit und spezifizieren Maßstäbe, welche zur Laufzeit der Anwendung eingehalten werden müssen. Darunter gehören beispielsweise die effiziente Ressourcennutzung, die Korrektheit des Systems sowie ein gewisser Grad an Zuverlässigkeit.
\add{09.06}

\add{Nichtfunktionale Anforderungen als Unterpunkte}
\subsubsection{...}

\section{Konzept und Modellierung}
\label{konzept_und_modellierung}
\add{10.06-13.06}
\add{Intro}

\subsection{Designentscheidungen}
\label{konzept_und_modellierung_designentscheidungen}

\subsection{...}
\chapter{Implementierung}
\label{implementierung}

\add{Implementierung}

\section{Architektur}
\label{architektur}

\add{Architektur}

\subsection{...}

\section{...}
\add{Extend}
\chapter{Evaluation}
\label{evaluation}
Im folgenden Kapitel werden Kernkonzepte von \texttt{SimpliFX} durch Quelltextbeispiele und eine Beispielanwendung erklärt. Außerdem wird eine Applikation erstellt, welche äquivalente Funktionalitäten wie die Beispielanwendung aufweist, aber vollständig auf die Nutzung von \texttt{SimpliFX} verzichtet. Beide Anwendungen werden in verschiedenen Aspekten wie der Benutzerfreundlichkeit oder dem Zeitaufwand verglichen und die Ergebnisse werden abschließend in einem Fazit zusammengefasst. 
\add{Intro}

\section{Entwicklung von Beispielsoftware}
\label{entwicklung_von_beispielsoftware}
%\subsection{Voraussetzungen}
\add{maybe add comment}
\noindent Bevor die Entwicklung der eigentlichen Software begonnen werden kann, müssen etwaige externe Bibliotheken wie JavaFX für \texttt{SimpliFX} bereitgestellt werden. Außerdem muss \texttt{SimpliFX} zur Kompilierzeit im Klassenpfad der zu entwickelnden Anwendung existieren. Da die Bibliothek im Github Maven-Repository verfügbar ist, wird der folgende Entwicklungsprozess sowie die Verwaltung von externen Bibliotheken durch die Verwendung von Maven unterstützt. In der \texttt{pom.xml} müssen für eine volle Funktionalität folgende Artefakte als Abhängigkeiten deklariert werden:
\begin{itemize}
	\item \texttt{de.intelligence:simplifx-guice} für das Nutzen aller Basisfunktionen von \texttt{SimpliFX} und der Kompatibilität zu Guice für die Abhängigkeitsinjektion.
	\item \texttt{org.openjfx:javafx-controls:16} für Kontrollkomponenten wie z.B. Schaltflächen.
	\item \texttt{org.openjfx:javafx-fxml:16} für das Verwenden von FXML Dateien.
	\item \texttt{org.openjfx:javafx-graphics:16} für das Darstellen von Komponenten im Szenengraph. Außerdem muss bei der Artefaktdeklaration die jeweilige Zielplattform als \texttt{classifier} angegeben werden. Ein Beispiel für die Windows-Plattform ist nachfolgend dargestellt.
	\begin{lstlisting}[language=XML, frame=none, belowskip=0pt]
<dependency>
    <groupId>org.openjfx</groupId>
    <artifactId>javafx-graphics</artifactId>
    <version>16</version>
    <classifier>win</classifier>
</dependency>	
	\end{lstlisting}
	\item \texttt{com.jfoenix:jfoenix:9.0.10} für erweiterte Designkomponenten.\footnote{Die neuste Version von JFoenix ist aufgrund der Jigsaw-API nicht direkt mit Java 16 kompatibel. Um die Bibliothek dennoch zu verwenden, werden eventuelle, auf das Modularitätssystem zurückführbare, Probleme durch das Nutzen der Reflection-Schnittstelle gelöst.}
\end{itemize}
Die vollständige \texttt{pom.xml} sowie alle in diesem Unterkapitel erstellten Klassen und Dateien sind im öffentlichen Github Maven-Repository unter dem Artefakt \texttt{de.intelligence:demo-applications} einsehbar.
\subsection{Struktur und Funktion der Beispielanwendung}
Die Anwendung soll weitgehend alle Funktionen von \texttt{SimpliFX} in Anspruch nehmen. Um die Konfigurationsschnittstelle sowie die Abhängigkeitsinjektion demonstrativ zu zeigen, wird eine Konfigurationsdatei definiert, welche Zugangsdaten und Verbindungsparameter zu einem imaginären Server enthält. Dazu wird ein Service erstellt, welcher für die Behandlung des Loginprozesses verantwortlich ist und durch ein Guice Modul zur Verfügung gestellt wird. Für die dynamische Lokalisierungsfunktion wird ein \texttt{ResourceBundle} mit dem Basisnamen \texttt{Messages} in sowohl der deutschen als auch der englischen Sprache erstellt. Wenn die Anwendung gestartet wird, soll eine grafische Schnittstelle für ein exemplarisches Einloggen angezeigt werden, welche Standardfunktionen wie das Eingeben der Zugangsdaten sowie die Änderung der Standardsprache zulassen soll. Die Funktion zur Sprachänderung kann dabei durch eine JavaFX \texttt{MenuBar} erfolgen. Wenn ein eventueller Login erfolgreich war, soll dem Entwickler eine Seitenleiste sowie ein Bereich, dessen Inhalt durch ebendiese kontrolliert wird, präsentiert werden. Die Seitenleiste soll vier Schaltflächen zur Navigation durch die Testapplikation, sowie ein Textfeld, welches die Verbindungsdaten zum Server anzeigt, enthalten. Der Startcontroller wird im folgenden als \texttt{MainController} bezeichnet. 
Dieser Controller hat als Wurzelelement eine \texttt{BorderPane} und erstellt zwei Controller Untergruppen in dieser. Im oberen Bereich wird die \texttt{titleBar} Gruppe initialisiert, welche Anwendungseinstellungen unabhängig vom aktuell angezeigten Controller bereitstellt und im Mittelbereich die \texttt{mainContent} Gruppe, welche die Darstellung der eigentlichen Anwendung übernimmt und beim Start den \texttt{LoginController} als aktiven Controller anzeigt. Die \texttt{mainContent} Gruppe beinhaltet außerdem den \texttt{MainMenuController}, welche wiederum eine linke Seitenleiste sowie einen Bereich für andere Inhalte neben dieser verwaltet. Eine Übersicht aller verwendeten Controller und Controllergruppen ist in \autoref{fig:controller_relations} abgebildet. Blaue Kreise sind dabei Gruppen und Linien welche einen oder mehrere Controller verbinden, spezifizieren das Subcontroller Verhalten. Wenn beispielsweise der \texttt{MainMenuController} der aktive Controller in der \texttt{mainContent} Gruppe ist und zum \texttt{LoginController} gewechselt wird, so werden auch entsprechende Lebenszyklusmethoden im aktiven Controller von \texttt{sidebarContent} und \texttt{sidebar} aufgerufen.
\begin{figure}[H]
	\centering
	\includegraphics[width=\textwidth]{Abbildungen/Controller Relations.png}
	\caption{Diagramm -- Controller und Controllergruppen.}
	\label{fig:controller_relations}
\end{figure}
\subsection{Implementierung der Beispielanwendung}
Damit eine \texttt{SimpliFX} Anwendung als solche erkannt wird, muss eine Klasse definiert werden, welche als Einstiegspunkt dienen soll. Diese muss die Annotation \texttt{@ApplicationEntryPoint} aufweisen und als Parameter den Startcontroller (\texttt{MainController}) übergeben. Für die Abhängigkeitsinjektion mit Guice muss der Einstiegspunkt auch mit \texttt{@GuiceInjection} annotiert werden und die Klasse eines Guice-Moduls übergeben. Wie in der Einleitung bereits beschrieben, wird das Modul nur die Implementierung des Login Services bereitstellen (siehe \autoref{fig:login_service}). 
\begin{figure}[H]
	\centering
	\includegraphics[width=\textwidth-5cm]{Abbildungen/Login Service.png}
	\caption{Diagramm -- Login Service.}
	\label{fig:login_service}
\end{figure}
\add{Entwicklung von Beispielsoftware}

\section{Vergleich konventioneller Methoden mit entwickeltem System}
\label{vergleich_system_javafx}

\add{Vergleich konventioneller Methoden mit entwickeltem System}

\chapter{Fazit}
\label{fazit}

\add{Intro}

\section{Zusammenfassung}
\label{zusammenfassung}

\add{Zusammenfassung}

\section{Bewertung}
\label{bewertung}

\add{Bewertung}

\section{Ausblick und mögliche Erweiterungen}
\label{ausblick_und_mögliche_erweiterungen}

\add{Ausblick und mögliche Erweiterungen}

\begin{appendix}
\chapter{Appendix 1}
\chapter{Appendix 2}
\end{appendix}

% Abkürzungsverzeichnis
\chapter*{Abkürzungsverzeichnis}
\label{abkürzungsverzeichnis}

\addcontentsline{toc}{chapter}{Abkürzungsverzeichnis}
\add{order alphanumerical}
\add{maybe switch to the acronym package for automatic sorting}
\begin{acronym}
	\acro{mvc}[MVC]{Model-View-Controller}
	\acro{xml}[XML]{Extensible Markup Language}
	\acro{loc}[LoC]{Lines of Code}
	\acro{css}[CSS]{Cascading Style Sheets}
	\acro{uml}[UML]{Unified Modeling Language}
	\acro{soc}[SoC]{Separation of Concerns}
\end{acronym}



% Abbildungsverzeichnis
\listoffigures

% Tabellenverzeichnis
\listoftables

% Literaturverzeichnis...
\nocite{*}
\bibliographystyle{geralpha}
\bibliography{Bibliographie/Bibliographie}

\listoftodos[Notes]

% Index ausgeben
\cleardoublepage
\addcontentsline{toc}{chapter}{Index}
\printindex

\end{document}
