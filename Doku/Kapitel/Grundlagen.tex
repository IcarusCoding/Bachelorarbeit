\chapter{Grundlagen}
\label{grundlagen}
\add{Correction}
\noindent In diesem Kapitel werden die theoretischen Grundlagen von essentiellen Komponenten dieser Arbeit erläutert. Dazu wird die Relevanz von Entwurfsmustern justifiziert und auf zwei bedeutende Muster näher eingegangen. Diese sind sowohl erforderlich für die folgenden Kapitel als auch für das Verständnis der softwaretechnischen Prinzipien von JavaFX.\\
Danach wird die JavaFX-Bibliothek vorgestellt und fundamentale Konzepte wie beispielsweise die auf der \ac{xml} basierende Layouting-Sprache erläutert.\\
Abschließend wird das generelle Annotationenkonzept in der Informatik mit speziellen Fokus auf die Programmiersprache Java erklärt. Dabei werden die verschiedenen Annotationstypen näher beschrieben und jeweils mit Beispielen untermauert, sowie die Möglichkeiten der eigentlichen Auswertung von Annotationen skizziert.

\section{Entwurfsmuster}
\label{entwurfsmuster}

\add{Intro}

\subsection{Definition}
\label{entwurfsmuster_definition}

\add{Definition Entwurfsmuster}

\subsection{Notwendigkeit}
\label{entwurfsmuster_notwendigkeit}

\add{Notwendigkeit \& Justifikation von Entwurfsmustern}

\section{JavaFX}
\label{javafx}
\add{Intro}

\subsection{Funktionsumfang}
\label{javafx_funktionsumfang}

\add{Funktionsumfang JavaFX}

\section{Java-Annotationen}
\label{java_annotationen}
Annotationen sind in der Sprachwissenschaft eine Möglichkeit einen vorhandenen Text mit Anmerkungen zu versehen für beispielsweise Disambiguierung, also das Eliminieren von Mehrdeutichkeiten eines Wortes oder für das Erklären von komplexen Textabschnitten. Sie geben dem Leser Zusatzinformationen um Sachverhalte einfacher darzustellen und sorgen dadurch für ein schnelleres bzw. besseres Verständnis des Textes. Dabei sind solche Anmerkungen kein Hauptbestandteil von Texten sondern dienen ausschließlich als Ergänzung.\\
In der Informatik sind Annotationen ebenfalls nur ein deskriptives Strukturkonzept, welche es dem Entwickler ermöglicht, verschiedenen strukturellen Elementen der Programmierung (wie Felder oder Klassen), Metadaten zuzuweisen \cite{Yu2019}. Das Nutzen von Annotationen in Anwendungen ist aufgrund ihrer meist simpel gehaltenen Syntax auch für Programmiereinsteiger vorteilhaft und durch ihre Anpassungsfähigkeit und Flexibilität sind sie in vielen Bibliotheken und Programmiersprachen vertreten.
\subsection{Definition}
\label{java_annotationen_definition}
\add{Reference}
\add{Move footnote to first occurence}
\noindent Annotationen \footnote{Wenn in der Arbeit über Annotationen gesprochen wird, ist immer von Java-Annotationen auszugehen (außer anders angegeben)} wurden mit Java 5 (2014) in die Sprache eingeführt und werden seitdem immer häufiger für verschiedene Aspekte der Programmierung genutzt \cite{Rocha2011}. Mit ihnen kann eine Steuerung des Compilers erfolgen, eine Verarbeitung der Metadaten zu Kompilierzeit durchgeführt werden oder das Verhalten von Anwendungen zu Laufzeit modifiziert oder gelenkt werden \cite{Yu2019}. Aufgrund der Tatsache, dass es sich nur um rein deskriptive Metadaten handelt, ist es Annotationen nicht direkt möglich mit existierendem Quelltext zu interagieren.  Möglichkeiten zur Verarbeitung dieser Metadaten werden in Kapitel \ref{TODO - Auswertung Ref} vorgestellt. Neben den von Java vordefinierten Annotationen wie z.B. \texttt{@Override} für das Überschreiben von vererbten Methoden oder \texttt{@SuppressWarnings} für das Unterdrücken von Compilerwarnungen, können auch eigene Annotationen deklariert werden.\\
Es handelt sich bei Annotationen in Java um spezialisierte Schnittstellen bei welchen das \texttt{interface}-Schlüsselwort durch ein \texttt{@}-Zeichen Präfix zu \texttt{@interface} erweitert wird \cite{Gosling2005}. Außerdem ist es Annotationen nicht erlaubt wie bei normalen Schnittstellendefinitionen das Schlüsselwort \texttt{extends} für eine Vererbung zu verwenden, da die Superschnittstelle implizit vom Compiler auf die \texttt{Annotation} Klasse des \texttt{java.lang.annotation} Pakets gesetzt wird \cite{Oracle2017}. Ein Beispiel einer  Annotationsdefinition ist in \autoref{lst:annotation_definition} dargestellt.
\begin{figure}[H]
	\centering
	\begin{lstlisting}[caption={Beispiel einer Annotationsdefinition.}, captionpos=b, label=lst:annotation_definition]
	public @interface TestAnnotation {
	    // ...
	}
	\end{lstlisting}
\end{figure}
\noindent In der Analogie des Kapitels \ref{java_annotationen} können Elemente mit strukturgebenden Charakter wie Bestandteile eines Satzes annotiert werden. Analog dazu sind in der Java-Programmierung Klassen, Methoden, Felder etc. für die Strukturierung des Quelltextes und der Softwarearchitektur verantwortlich und somit auch mit Annotationen erweiterbar. Um Sprachelemente zu annotieren muss wie in \autoref{lst:annotated_example} dargestellt, ein \texttt{@}-Präfix zum eigentlichen Klassennamen hinzugefügt werden.
\begin{figure}[H]
	\centering
	\begin{lstlisting}[caption={Beispiel einer annotierten Klasse.}, captionpos=b, label=lst:annotated_example]
	#@TestAnnotation
	public class TestClass {
	    // ...
	}
	\end{lstlisting}
\end{figure}
\noindent Aufgrund der besonders einfachen Syntax und dem vergleichsweise geringen Aufwand, ist ein steigender Trend der Nutzung von Java-Annotationen in Open-Source Anwendungen zu erkennen. Werden Annotationen jedoch übermäßig verwendet, so kann es schnell zu Quelltext-Verschmutzung kommen, was im Kontext der Annotationsprogrammierung auch \glqq annotation hell\grqq{} (dt. Annotationshölle) genannt wird. Annotationen erreichen dann das Gegenteil des gewünschten Zwecks -- Statt den Entwicklungsprozess vereinfachend zu unterstützen, wird der Quelltext schwer nachvollziehbar und wirkt unstrukturiert und unübersichtlich.\\
Dennoch zeigt eine Studie aus dem Jahre 2011, welche 1094 quelloffene GitHub-Projekte auf die Verwendung von Annotationen untersucht hat, dass javabasierte Anwendungen und Bibliotheken, bei aktiver Nutzung von Annotationen, eine geringere Fehleranfälligkeit aufweisen \cite{Rocha2011}.
\subsection{Syntax}
\label{java_annotationen_anwendung}
\add{lst design}
\noindent Annotationen können Attribute besitzen, welche bei Kompilierzeit bzw. Laufzeit ausgelesen werden können. Die Typen dieser Attribute sind nicht vollständig frei wählbar -- So ist es beispielsweise nicht möglich ein Attribut vom Typen \texttt{Object} in einer Annotation zu kapseln, ohne einen Kompilierfehler auszulösen. Erlaubt sind alle primitiven bzw. atomaren Datentypen und Instanzen der \texttt{String}-, \texttt{Class}- und \texttt{Enum}-Klasse sowie eindimensionale Arrays aus den vorherigen Typen. Außerdem ist es möglich, Attributen einen voreingestellten Wert mittels des Schlüsselwortes \texttt{default} zuzuweisen \cite{Gosling2005}. Annotationen müssen in einer der folgenden Syntaxen benutzt werden:
\begin{description}
	\item \textbf{Normal Annotations} sind ganz normal deklarierte Annotationen, bei welchen die Attribute mittels Aufzählung in Klammern übergeben werden.
	\begin{figure}[H]
		\noindent
		\newlength\heightone
		\begin{adjustbox}{minipage=[t]{.45\linewidth},gstore totalheight=\heightone,margin=\fboxsep+\fboxrule}
			\begin{lstlisting}[caption=Deklaration -- Normal Annotation., captionpos=b, label=lst:decl_normal]
public @interface Entity {
	String name();
	int id();
}
			\end{lstlisting}
		\end{adjustbox}\hfill
		\begin{adjustbox}{minipage=[t][\heightone]{0.5\linewidth}}
			\begin{lstlisting}[caption=Anwendung -- Normal Annotation, captionpos=b, label=lst:appl_normal]
#@Entity(name="test", id=2)
public class TestEntity {
	// ...
}
			\end{lstlisting}
		\end{adjustbox}
	\end{figure}
	\item \textbf{Single-Element Annotations} sind eine Kurzform der normalen Annotationen mit einem \texttt{value}-Attribut und keinen weiteren nicht-default Attributen.
	\begin{figure}[H]
		\noindent
		\begin{adjustbox}{minipage=[t]{.45\linewidth},gstore totalheight=\heightone,margin=\fboxsep+\fboxrule}
			\begin{lstlisting}[caption=Deklaration -- Single-Element Annotation., captionpos=b, label=lst:decl_single]
public @interface Entity {
	String value();
	int id() default -1;
}
			\end{lstlisting}
		\end{adjustbox}\hfill
		\begin{adjustbox}{minipage=[t][\heightone]{0.5\linewidth}}
			\begin{lstlisting}[caption=Anwendung -- Single-Element Annotation, captionpos=b, label=lst:appl_single]
#@Entity("test")
public class TestEntity {
	// ...
}
			\end{lstlisting}
		\end{adjustbox}
	\end{figure}
	\item \textbf{Marker Annotations} sind ebenfalls eine Kurzform der normalen Annotationen mit keinen oder nur default Attributen.
	\begin{figure}[H]
		\noindent
		\begin{adjustbox}{minipage=[t]{.45\linewidth},gstore totalheight=\heightone,margin=\fboxsep+\fboxrule}
			\begin{lstlisting}[caption=Deklaration -- Marker Annotation., captionpos=b, label=lst:decl_marker]
public @interface Entity {
	String name() default "";
	int id() default -1;
}
			\end{lstlisting}
		\end{adjustbox}\hfill
		\begin{adjustbox}{minipage=[t][\heightone]{0.5\linewidth}}
			\begin{lstlisting}[caption=Anwendung -- Marker Annotation, captionpos=b, label=lst:appl_marker]
#@Entity
public class TestEntity {
	// ...
}
			\end{lstlisting}
		\end{adjustbox}
	\end{figure}
\end{description}
\noindent Die Sichtbarkeit von eigenen Annotationen zu verschiedenen Phasen des Codezyklus kann durch die von Java bereitgestellte Annotation \texttt{@Retention} gesteuert werden. Das übergebene Enum-Attribut klassifiziert die Annotation dann in einen von drei Typen \cite{Rocha2011}:
\begin{description}
	\item \textbf{Quellcode-Annotationen} sind nur beim Kompiliervorgang auslesbar und können dem Compiler Anweisungen geben oder mithilfe von Annotation-Prozessoren z.B. neue Klassen automatisch generieren. Sie sind in der kompilierten Java-Anwendung nicht mehr erhalten.
	\item \textbf{Klassen-Annotationen} sind nach dem Kompilierungsprozess noch in der Anwendung erhalten und können durch externe Tools wie z.B. dem Code-Obfuskator ProGuard ausgelesen werden.
	\item \textbf{Laufzeit-Annotationen} sind nach der Kompilierung und beim Start der Anwendung erhalten und können dann mithilfe der Reflection-API zur Laufzeit ausgewertet werden.
\end{description}
\noindent Des Weiteren kann gesteuert werden, welche Typen der Strukturelemente eines Quellcodes annotiert werden können. Ein Beispiel für eine zur Laufzeit beibehaltene Annotation, welche nur an Methoden angebracht werden kann ist in \autoref{lst:full_annotation_example} zu erkennen.
\begin{figure}[H]
	\centering
	\begin{lstlisting}[caption={Beispiel einer Laufzeit Annotation.}, captionpos=b, label=lst:full_annotation_example]
	
	#@Target(ElementType.(@\tikzmark{aLeft}{}@)METHOD(@\tikzmark{aRight}{}@))
	#@Retention(RetentionPolicy.(@\tikzmark{bLeft}{}@)RUNTIME(@\tikzmark{bRight}{}@))
	public @interface Event {
		int id();
		int priority() default 0;
	}
	(@
	\begin{tikzpicture}[overlay,remember picture]
		\foreach \x/\y in {a/red, b/blue} {
			\DrawOverBar[-, \y, thick]{\x Left.north}{\x Right.north}
		}
		\node[draw](onlymethods) at (8.5,3) {Nur an Methoden};
		\node[draw](runtime) at (8.5,1.5) {Zur Laufzeit};
		\DrawArrow[red, in=-180]{a}{onlymethods}{-4.8em, 0}
		\DrawArrow[blue, in=-270, out=20]{b}{runtime}{0, 0.65em}
	\end{tikzpicture}
	@)
	\end{lstlisting}
\end{figure}

\add{Add compile time annotation processing if used in this thesis}
\subsection{Auswertung von Laufzeit-Annotationen}
Für eine Auswertung von Laufzeit-Annotationen, muss zwangsläufig die Reflection-API von Java genutzt werden. Wenn eine Programmiersprache eine Form von Reflection (dt. Spiegelung) aufweist, so ist es möglich Attribute, Logikfluss und andere Eigenschaften während der Laufzeit zu ändern. In objektorientierten Sprachen wie Java wird diese \glqq computational reflection\grqq{} genutzt, um die Möglichkeit einer Selbstbeobachtung der eigenen Sprachelemente zu schaffen \cite{Li2017}. Die API ermöglicht somit beispielsweise das Auslesen von Laufzeit-Annotationen und deren deklarierte Attribute oder das dynamische Instanziieren von Klassen \cite{Forman2004}. Jedes Java-Element der Reflection API (Feld, Methode, Klasse, ...), welches annotierbar ist, wird durch die Vererbung der \texttt{AnnotatedElement}-Klasse als solches klassifiziert \cite{Schildt2019}. Damit nun alle vorhandenen Annotation ausgelesen werden können, kann die Methode \inlinecode{java}{AnnotatedElement#getDeclaredAnnotations} aufgerufen werden \cite{Pigula2015}. Das Lesen der Attribute der in \autoref{lst:full_annotation_example} vordefinierten Annotation ist in \autoref{lst:annotation_processing_example} zu erkennen.
\begin{figure}[H]
	\begin{lstlisting}[caption={Auslesen einer Laufzeit-Annotation.}, captionpos=b, label=lst:annotation_processing_example]
    if(Test.class.isAnnotationPresent(Event.class)) {
	    Event e = Test.class.getDeclaredAnnotation(Event.class);
	    int id = e.id();
	    int priority = e.priority();
    }
	\end{lstlisting}
\end{figure}

\subsection{Beispiele der Annotationsprogrammierung}
\label{java_annotationen_annotationsprogrammierung}

\add{Beispiele der Annotationsprogrammierung}