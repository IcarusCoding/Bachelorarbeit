\chapter{Stand der Technik}
\label{stand_der_technik}
Im diesem Kapitel werden aktuelle Konzepte und Implementierungen der Annotationsprogrammierung zur Vereinfachung des Entwicklungsprozesses einer Anwendung dargelegt. Obwohl der primäre Fokus dabei auf der JavaFX- und der generellen Java-Umgebung gelegt wird, werden dennoch auch Bibliotheken und mögliche Strukturen aus anderen Programmiersprachen herangezogen.
\add{complete intro}

\section{Aktuelle Verwendung von Annotationen}
\label{aktuelle_verwendung_von_annotationen}
\add{Intro wie in github wiki}
\add{structure}

\subsection{Annotationen in anderen Programmiersprachen}
Neben den Java-Annotationen, welche in \autoref{java_annotationen} vorgestellt und erklärt wurden, werden Annotationen auch in vielen anderen Programmiersprachen genutzt. 
\subsubsection{Verwendung in Python}
\label{verwendung_in_python}
Python ist eine dynamisch typisierte Sprache und validiert somit den Typen einer Variablen zur Laufzeit des Programms \cite{Tratt2009}, kann aber durch das Verwenden von Funktionsannotationen, Meta-Daten zu Parametern, Variablen und Funktionsrückgabewerten hinzufügen, um so den gewünschten Typen anzudeuten \cite{Rossum2014, Winter2006}. Diese Annotationen werden zwar vom Python-Interpreter ignoriert, können aber durch Softwaresysteme von Drittanbietern wie \texttt{mypy} zur statischen Typisierung verwendet werden. Nach einer Studie von Khan et al., welche 210 auf Python basierende GitHub-Projekte auf typbezogene Fehler untersuchte, konnten 15\% der gefundenen Mängel, durch \texttt{mypy} verhindert werden \cite{Khan2021}.
Einige Entwicklungsumgebungen wie PyCharm sind außerdem in der Lage, Warnungen bei eventuellen Verletzungen der Typempfehlungen von Annotationen anzuzeigen \cite{Rother2017}. Das Verwenden von derartigen Annotationen kann somit durchaus die Fehleranfälligkeit von Programmcodeelementen sinken -- wenn auch nur implizit durch externe Bibliotheken oder Entwicklungsumgebungen.
\add{maybe add code example}
\subsubsection{Verwendung in \texorpdfstring{\csharpbold}{C\#} und .NET}
In \csharp ... \cite{Albahari2019}
\label{verwendung_in_c_sharp_dot_net}
\subsection{JavaFX}
\label{aktuelle_verwendung_von_annotationen_javafx}

\add{JavaFX Beispiele}

\subsection{Android}
\label{aktuelle_verwendung_von_annotationen_android}

\add{Android Beispiele}

\subsection{JavaX}
\label{aktuelle_verwendung_von_annotationen_javax}

\add{JavaX Beispiele (z.B. JAXB)}

\section{Maßnahmen zur Simplifizierung des Entwicklungsprozesses}
\label{maßnahmen_zur_simplifizierung_des_entwicklungsprozesses}

\add{Intro}

\subsection{Workflow Optimierung}
\label{maßnahmen_zur_simplifizierung_des_entwicklungsprozesses_workflow}

\add{Workflow Optimierung}

\subsection{Vereinfachung durch gesteigerte Übersichtlichkeit}
\label{maßnahmen_zur_simplifizierung_des_entwicklungsprozesses_übersichtichkeit}

\add{Vereinfachung durch gesteigerte Übersichtlichkeit}

\subsection{Fazit}
\label{maßnahmen_zur_simplifizierung_des_entwicklungsprozesses_fazit}

\add{Fazit}