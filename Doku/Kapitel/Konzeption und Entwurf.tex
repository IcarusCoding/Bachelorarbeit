\chapter{Konzeption und Entwurf}
\label{konzeption_und_entwurf}
In diesem Kapitel werden mögliche Probleme bei der Entwicklung sowie bei der Nutzung von JavaFX Anwendungen identifiziert. Dabei wird ein besonderer Fokus auf das Finden von Architekturmängeln, fehlenden Funktionalitäten und verbesserungswürdigen Techniken gelegt. Um eine Fehleranfälligkeit zu reduzieren, sollen komplexe und sich häufig wiederholende Quelltextbausteine automatisch erstellt oder durch Annotationen vereinfacht werden. Die vollständige Substitution eines aufwendigen Prozesses ist dabei ebenfalls möglich. Probleme, Vereinfachungen oder Verbesserungen sollen durch das Untersuchen von vorhandenen, quelloffenen JavaFX-Projekten und Bibliotheken gefunden werden. Auch sollen Ideen und Konzepte zusammengetragen werden, welche auf JavaFX anwendbar sind, jedoch nur in anderen Bibliotheken und Frameworks aufzufinden sind. \\
Bei der Problemanalyse wird stets das Ziel verfolgt, das Entwickeln mit JavaFX zu vereinfachen -- besonders für noch unerfahrene Entwickler. Danach wird eine Anforderungsanalyse durchgeführt, mit welcher systematisch funktionale sowie nichtfunktionale Anforderungen auf der Basis der gefundenen Probleme erstellt werden. \\
Auf die Anforderungserhebung folgt die Konzeption des benötigten Systems und der zugrundeliegenden Architektur. Essentielle Komponenten werden mit \ac{uml} Diagrammen entworfen und im Detail erläutert. Bei der Existenz verschiedener Lösungsstrategien für ein Problem, wird jede Strategie einzeln beleuchtet und nach Kriterien wie Sinnhaftigkeit und Machbarkeit entschieden, welche für das System am besten geeignet ist. Wichtige Richtlinien wie die angestrebte Softwarequalität werden ebenfalls beschrieben.

\section{Identifikation von Problemen und komplexen Strukturen in der JavaFX Entwicklung}
\sectionmark{Identifikation von Problemen}
\label{problemanalyse}
Im Folgenden werden generelle Probleme bei der Entwicklung von JavaFX Anwendungen identifiziert. Dazu gehören Mechanismen welche aufgrund ihrer Komplexität nicht für Anfänger geeignet sind oder von erfahrenen Entwicklern häufig genutzt und somit möglicherweise vereinfacht werden können. Obwohl dabei Annotationen als Basis für eine Vereinfachung dienen, wird auch das Erstellen von zusätzlichen Klassen oder dem Entwickeln von Erweiterungen für existierenden JavaFX Konzepte als Alternative für diese Zielerreichung in Betracht gezogen. Die Lösungen der gefundenen Probleme werden in einer Anforderungsanalyse durch funktionale und nichtfunktionale Anforderungen in \autoref{anforderungsanalyse} gelöst.
\subsection{Internationalisierung und Lokalisierung}
In der Informatik, speziell in der Softwareentwicklung, ist die Internationalisierung ein wichtiger Bestandteil einen Softwareproduktes, bei welchen die Entwickler die Software so gestalten, dass diese ohne viel Aufwand für andere internationale Märkte mit anderen Kulturen und Sprachen verfügbar gemacht werden kann \cite{Reineke2005}. Dabei wird beispielsweise eine einfache Schnittstelle für das Verwenden von verschiedenen Sprachen entwickelt, welche das Übersetzen von vorhandenen Textfeldern und anderen textbasierte Elementen in grafischen Benutzeroberflächen, Konfigurationsdateien oder Konsolenausgaben ermöglicht. Die Schnittstelle wird dabei so entwickelt, dass ohne eine Änderung des Quelltextes, neue Sprachen hinzugefügt werden können. Die Lokalisierung beschreibt dann unter Anderem die Übersetzung von den eben genannten Elementen.\\
Dieses Konzept kann durch die von Java bereitgestellte \texttt{ResourceBundle} Klasse realisiert werden \cite{Deitsch2001}. Bei der Verwendung eines solchen \texttt{ResourceBundle}s wird jedem zu übersetzenden Element ein Schlüssel zugeordnet und nach Konvention, in einer \texttt{.properties} Datei gespeichert. Wenn eine neue Sprache im Laufe des Lokalisierungsprozesses hinzugefügt werden soll, so muss jeweils eine neue \texttt{.properties} Datei angelegt werden. JavaFX ermöglicht das manuelle Spezifizieren einer vordefinierten \texttt{ResourceBundle} Instanz bei dem Laden einer FXML Datei durch einen \texttt{FXMLLoader}.
In der zu ladenden FXML Datei müssen hartcodierte Textelemente durch den jeweiligen Schlüssel aus der \texttt{.properties} Datei, wie in \autoref{lst:fxmlkey} gezeigt, ersetzt werden.
\begin{figure}[H]
	\noindent
	\begin{adjustbox}{minipage=[t]{.45\linewidth},gstore totalheight=\heightone,margin=\fboxsep+\fboxrule}
		\begin{lstlisting}
## Properties Datei
login.user = Benutzername
		\end{lstlisting}
	\end{adjustbox}\hfill
	\begin{adjustbox}{minipage=[t][\heightone]{0.5\linewidth}}
		\begin{lstlisting}[language=XML]
<!-- FXML Datei -->
<Label text="%login.username"/>
		\end{lstlisting}
	\end{adjustbox}
	\captionof{lstlisting}{Nutzung des Schlüssels in einer FXML Datei.}
	\label{lst:fxmlkey}
\end{figure}
\noindent Das Problem bei dieser Art der Übersetzung ist, dass eine Änderung der Sprache zur Laufzeit des Programms nicht dynamisch möglich ist. Die FXML Datei bzw. der dazugehörige Controller muss nach einer Sprachänderung durch beispielsweise eine Schaltfläche oder ein Dropdown-Menü durch einen FXML-Loader neu geladen werden, damit eventuelle Änderungen übernommen werden können. Das dynamische Ändern der Sprache zur Laufzeit ist nur mit einem Modifizierung des Ladeprozesses von FXML Dateien durch eine eigene Version des \texttt{FXMLLoader}s oder durch eine vom \texttt{FXMLLoader} unabhängige Implementierung durch Properties und Bindings möglich. Die erste Variante sorgt für ein automatisches Binden der nötigen Properties und die Letztere für ein manuelles Binden wenn nötig, weshalb eine Fusion beider Möglichkeiten in eine hohe Anpassbarkeit des Systems resultiert. Außerdem ist es auf diese Weise möglich, verschiedene Bindings zu aktualisieren, falls ein bestimmtes Event auftritt, welches eine Änderung des übersetzten Textes hervorruft. Parameterisierte Schlüssel aus der \texttt{.properties} Datei können somit automatisch an die parameterverändernden Events gebunden werden.
\subsection{Abhängigkeitsinjektion für Controller}
Bei der Abhängigkeitsinjektion werden Abhängigkeiten von Objekten zur Laufzeit des Programms bestimmt und zur Verfügung gestellt. Meist konfiguriert der Entwickler mithilfe einer externen Bibliothek die Bereitstellung der Abhängigkeiten in einer Konfigurationsdatei oder Konfigurationsklasse. Bei dem Nutzen einer Art von Abhängigkeitsinjektion ist die eigentliche Implementierung der Abhängigkeiten durch die abhängigen Objekte nicht bekannt. Diese kennen nur die Schnittstellen, weshalb ein Auswechseln der Schnittstellenimplementierung durch eine Änderung der jeweiligen Konfigurationsdatei/Konfigurationsklasse möglich ist. Durch diese explizite Trennung von Schnittstelle und Implementierung ist eine lose Kopplung der Komponenten gewährleistet, was wiederum zu einer hohen Flexibilität und zu einer hohen Wartbarkeit sowie Testbarkeit führt. Eine Injektion ist dabei durch Felder (Feldinjektion), Konstruktoren (Konstruktorinjektion) oder Setter (Setterinjektion) möglich. Die Funktionalität einer Injektion von Abhängigkeiten in einen Controller ist nicht direkt in JavaFX enthalten, kann aber durch das Verwenden von zum Beispiel Afterburner.fx\footref{ft:afterburner} hinzugefügt werden. Eine Unterstützung von etablierten externen Bibliotheken zur Realisierung des Abhängigkeitsinjektionsmusters wie Spring\footnote{Spring: \url{https://spring.io}}, Dagger\footnote{Dagger: \url{https://dagger.dev}} oder Guice\footnote{\label{ft:guice}Guice: \url{https://github.com/google/guice}} ist dadurch jedoch nicht gegeben. Aufgrund der Tatsache, dass alle drei genannten Bibliotheken mit der \texttt{jakarta.inject.Inject} Annotation\add{validate}, eine grundlegende Abhängigkeitsinjektion bereitstellen, kann ebendiese Annotation für eine Injektion innerhalb von Controllern verwendet werden. Ein Beispiel eines Controllers mit verschiedenen injizierten Diensten ist in \autoref{lst:controllerinjection} zu erkennen.
\begin{figure}[H]
	\centering
	\begin{lstlisting}[caption=Beispiel -- Controller mit injizierten Diensten., captionpos=b, label=lst:controllerinjection]
	public class TestController {
	
		#@Inject
		private IUserService userService;
		
		#@Inject
		private IDatabaseService dbService;
	
	}
	\end{lstlisting}
\end{figure}
\noindent Die Implementierungen sollen durch den Entwickler in der jeweiligen Konfigurationsdatei/Konfigurationsklasse den Schnittstellen zugeordnet werden können.
\subsection{CSS Metadatengeneration}
Ein wichtiger Bestandteil von JavaFX ist die Möglichkeit, das Aussehen und den Stil von einzelnen Komponenten wie Schaltflächen und Containerelementen durch die Verwendung von \ac{css} zu modifizieren. Werden eigene Komponenten erstellt oder bestehende Komponenten erweitert, so erlaubt JavaFX das Hinzufügen von eigenen \ac{css} Properties zu den jeweiligen Komponenten mithilfe von neuen Instanzen der \texttt{CssMetaData} Klasse. Für jede neue \ac{css} Property muss dabei eine neue \texttt{CssMetaData} Instanz erstellt werden, weshalb das Hinzufügen von vielen solcher Instanzen ein repetitiver Prozess mit vielen Boilerplate Quelltextfragmenten ist, welcher durchaus vereinfacht werden kann.
\subsection{JavaFX Einstiegspunkt und Preloader}
Damit eine JavaFX Applikation gestartet werden kann muss, wie in \autoref{javafx} beschrieben, 
eine Klasse existieren, welche von der \texttt{Application} Klasse erbt. Müssen bestimmte performanceintensive Aufgaben wie das Laden von Sound-/Video- oder Bilddateien vor dem Start der eigentlichen Anwendungen ausgeführt werden, so kann dies in einer Klasse, welche von der \texttt{Preloader} Klasse erben muss, realisiert werden. Der \texttt{Preloader} ist dabei eine spezialisierte Form der \texttt{Application}, welche es dem Entwickler ermöglicht, Ressourcen zu laden und durch eventbasierte Statusaktualisierungen, dem Nutzer mitzuteilen. Damit ein \texttt{Preloader} mit einer standardmäßigen JavaFX Anwendung verbunden werden kann, muss statt \texttt{Platform\#launch}, die interne \texttt{PlatformImpl\#launchApplication} Methode genutzt werden. Das Verwalten der Aufgaben, welche vor dem Anwendungsstart durchgeführt werden müssen, soll vereinfacht werden, wobei insbesondere die vorhandenen Statusaktualisierungen  erweitert werden sollen. Auch die Initialisierung einer Applikationsklasse bzw. der benötigten \texttt{Stage} läuft in den meisten Fällen gleich ab und soll bei Bedarf vom Entwickler zu einem großen Teil automatisiert werden können.
\subsection{Controller Lebenszyklus}
Der Lebenszyklus von JavaFX Controllern ist in der aktuellen Ausführung für komplexe Systeme mit vielen Controllern ungeeignet, da dieser nur aus zwei Phasen besteht. Zuerst wird die Controllerklasse instanziiert und darauf folgt die Initialisierung der \texttt{@FXML} Felder und der Methodenaufruf einer vom Entwickler bereitgestellten \texttt{initialize} Methode. Damit mit fertig initialisierten \texttt{@FXML} Feldern gearbeitet werden kann, muss der dafür benötigte Quelltext immer in der \texttt{initialize} Methode definiert werden. Ein Beispiel für den Ablauf einer Controllerinstanziierung durch den \texttt{FXMLLoader} ist in \autoref{lst:controllerinstantiation} abgebildet. Der rudimentäre Lebenszyklus unterstützt dabei keine Methoden, welche beispielsweise bei einem Entfernen des Wurzelelements eines Controllers aus dem Szenengraphen ausgeführt wird um eventuell offene Ressourcen zu schließen und somit effektiv Ressourcenlecks zu verhindern. Ein vollständiger Lebenszyklus wie bei Activities in Android\footnote{https://developer.android.com/guide/components/activities/activity-lifecycle},ist praktisch nicht vorhanden.
\begin{figure}[H]
	\centering
	\begin{lstlisting}[caption=Beispiel -- Instanziierungsablauf., captionpos=b, label=lst:controllerinstantiation]
public class TestController {
	
	@FXML
	private Label testLbl;
	
	public TestController() {
		// Phase #1 - Controllerinstanziierung
		// testLbl-Feld ist hier noch nicht initialisiert
	}
	
	@FXML
	private void initialize() {
		// Phase #2 - Feldinjektion fertiggestellt
		// testLbl-Feld ist initialisiert und kann verwendet werden
	}
	
}
	\end{lstlisting}
\end{figure}
\section{Anforderungsanalyse}
\label{anforderungsanalyse}
In der Anforderungsanalyse werden die gefundenen Problemlösungen und Vereinfachungen aus \autoref{problemanalyse} in Form von funktionalen und nichtfunktionalen Anforderungen formuliert. Dabei werden die Anforderungen in zwei Klassen unterteilt:
\begin{description}
	\item \textbf{Fundamentale Anforderungen} sind Anforderungen, welche für eine Funktion des Systems essentiell sind, alle genannten Probleme weitgehend beheben und daher zwangsläufig implementiert werden müssen. Alle folgenden fundamentalen Anforderungen haben den [FA\#\#] Prefix, wobei \#\# durch die jeweilige Anforderungsnummer substituiert wird.
	\item \textbf{Optionale Anforderungen} sind Anforderungen, welche keinen Einfluss auf eine ordnungsgemäße Funktionalität des Systems haben. Sie sind optional und werden möglicherweise aufgrund ihrer Komplexität nur teilweise oder gar nicht implementiert und können stattdessen für eine Erweiterung des Systems durch weitere Entwickler genutzt werden. Alle folgenden optionalen Anforderungen haben den [OA\#\#] Prefix, wobei \#\# durch die jeweilige Anforderungsnummer substituiert wird.
\end{description}
\add{format of requirements}

% counter for fundamental requirements
\newcounter{fcounter}
\setcounter{fcounter}{0}
\DeclareRobustCommand{\freq}{\stepcounter{fcounter}[FA\ifnum\value{fcounter}<10 0\fi\arabic{fcounter}]\label{fa\arabic{fcounter}}}

% counter for optional requirements
\newcounter{ocounter}
\setcounter{ocounter}{0}
\DeclareRobustCommand{\oreq}{\stepcounter{ocounter}[OA\ifnum\value{ocounter}<10 0\fi\arabic{ocounter}]\label{oa\arabic{ocounter}}}

\subsection{Funktionale Anforderungen}
\label{anforderungsanalyse_funktional}
Im Folgenden werden alle funktionalen Anforderungen definiert. Sie beschreiben alle gewünschten Funktionen des Endproduktes.
\subsubsection{\freq{} Ermitteln und Konfiguration des JavaFX-Einstiegspunktes}
Der Erstellen einer Applikationsklasse sowie eines optionalen Preloaders soll automatisch stattfinden, da der Erstellungsprozess von auf JavaFX basierten Anwendungen in den meisten Fällen ähnlich oder identisch ist. Das Definieren dieser Elemente soll mithilfe von Annotationen geschehen und durch das Absuchen des Klassenpfades durch die entwickelte Bibliothek gefunden werden. Alternativ soll der Entwickler die Möglichkeit haben, den Einstiegspunkt sowie den Preloader explizit zu spezifizieren, falls eine automatische Identifikation nicht erwünscht oder benötigt ist. Die vordefinierten Einstiegspunkte sollen eine Standardkonfiguration der Applikation und der zugrunde liegenden \texttt{Stage} Instanz bereitstellen, welche durch den Entwickler bei Bedarf auch manuell überschrieben werden kann. Dieses Überschreiben soll durch die Nutzung des Erbauer-Entwurfsmusters oder durch Annotationen erfolgen. Die Standardkonfiguration umfasst grundsätzliche Applikationseigenschaften wie den Titel, die Fenstergröße oder Designelemente wie den \texttt{StageStyle} und die benötigten Titelleistenschaltflächen.
\subsubsection{\freq{} Eventbasierte Verwaltung des Applikationslebenszyklus}
Die Verwaltung des Applikationslebenszyklus soll mithilfe von Annotationen im Einstiegspunkt geschehen. Der Applikationslebenszyklus bezeichnet hier die vererbbaren \texttt{Application\#init}, \texttt{Application\#start} und \texttt{Application\#stop} Methoden. Die Erstellung der Applikationsklasse soll wie in \autoref{TODO} beschrieben, durch die zu entwickelnde Bibliothek übernommen werden, weshalb die Aufrufe der eben genannten Methoden durch ein Event-System an den vom Entwickler spezifizierten Einstiegspunkt delegiert werden sollen. Beispielsweise wird statt der Überschreibung der \texttt{Application\#start} Methode, eine Methode im Einstiegspunkt definiert, welche ein \texttt{StartEvent} als Parameter erhält und eine spezielle Annotation aufweist. Methoden für einen optionalen Preloader, sollen analog dazu definiert werden können.
\add{Maybe add}
%\subsubsection{\freq{} Vordefinierte Preloader-Zustandsänderungsbenachrichtigungen}
\subsubsection{\freq{} Dynamische Laufzeitübersetzung von JavaFX-Komponenten}
Die bereits existierenden Internationalisierungsmöglichkeiten sollen durch eine dynamische Übersetzung erweitert werden. Das Ändern der aktuellen Sprache soll durch beispielsweise einen Schaltflächenklick ausgelöst werden und zur Laufzeit die grafische Benutzeroberfläche aktualisieren. Ein Neustart der Anwendung oder ein erneutes Laden der FXML Dateien soll nicht mehr nötig sein. Der Entwickler soll dabei JavaFX Komponenten in Form von Objektattributen in Controllern annotieren können, welche durch eine verbesserte Version der \texttt{FXMLLoader} Klasse geladen und instanziiert werden sollen. Das dynamische Übersetzen der in der FXML Datei definierten Elemente soll per Element deaktivierbar sein. Die Funktionalität von externen FXML Parsern wie den Scene-Builder soll durch das Hinzufügen von eigener FXML Syntax dabei nicht kompromittiert werden. Übersetzbare Elemente, welche auf parameterabhängigen Übersetzungsschlüsseln basieren, sollen an JavaFX Properties gebunden werden können und bei Änderung dieser aktualisiert werden. Wird eine dynamische Übersetzung für nicht annotierbare Prozesse wie das Ausgeben von Nachrichten auf der Konsole oder generelle methodeninterne Textverarbeitung benötigt, so soll der Entwickler auf Utility-Klassen der Bibliothek zurückgreifen können, welche ebendiese Funktionalität zur Verfügung stellen.
\subsubsection{\freq{} Abhängigkeitsinjektion durch etablierte externe Bibliotheken}
Werden externe Bibliotheken zur Abhängigkeitsinjektion genutzt, so sollen eventuelle Abhängigkeiten und Dienste automatisch in Controller injiziert werden. Die Nutzung etwaiger Bibliotheken und den dafür benötigten Konfigurationsdateien oder Konfigurationsklassen soll beim Start der Applikation spezifiziert werden können. Eine Unterstützung der Bibliotheken Dagger, Spring und Guice soll dem Entwickler ermöglicht werden.
\subsubsection{\freq{} Automatische \ac{css} Metadaten Generation}
Das Hinzufügen von \ac{css} Properties zu benutzerdefinierten JavaFX Komponenten durch den Entwickler soll durch das Nutzen von Annotationen erleichtert werden. Durch die automatische Generation dieser Metadaten, sollen eventuell auftretende repetitive Quelltextbausteine vermieden werden. Das Initialisieren der JavaFX Anwendung durch die zu entwickelnde Bibliothek soll in eine Durchsuchung des Klassenpfades nach solchen \ac{css} Metadaten Annotationen resultieren und die gefundenen Metadaten automatisch mit den durch die Annotation definierten Einstellungen instanziieren.
\subsubsection{\oreq{} Annotationsvalidierung zur Kompilierzeit}
Reflektive Operationen sind definitionsbedingt nur per Laufzeit der Anwendung ausführbar. Das Auslesen und Verarbeiten von annotierten Elementen und Annotationen im Allgemeinen durch reflektive Operationen kann somit nur zur Laufzeit erfolgen. Ob eine Annotation inkorrekte oder ungenügende Informationen enthält oder falsche Typen annotiert wurden ist dem Entwickler nur durch die Auslösung von Ausnahmen zur Laufzeit mitzuteilen. Beispielsweise ist es offensichtlich inkorrekt, ein Feld, welches einen JavaFX Container repräsentiert (z.B. \texttt{StackPane}), mit einer Annotation zu annotieren, welche für die Übersetzung von Komponenten verantwortlich ist. Das Überprüfen der Korrektheit der Annotationsnutzung soll nicht nur durch eine Laufzeitfehlerbehandlung sondern auch durch einen Annotationsprozessor zur Kompilierzeit durchgeführt werden. Eventuelle Fehlkonfigurationen werden somit bereits beim Kompiliervorgang entdeckt und dem Entwickler mitgeteilt.

\subsection{Nichtfunktionale Anforderungen}
\label{anforderungsanalyse_nichtfunktional}
Im Folgenden werden alle nichtfunktionalen Anforderungen definiert. Sie beschreiben Qualitätseigenschaften an das System wie Möglichkeiten der Erweiterbarkeit und Wartbarkeit und spezifizieren Maßstäbe, welche zur Laufzeit der Anwendung eingehalten werden müssen. Darunter gehören beispielsweise die effiziente Ressourcennutzung, die Korrektheit des Systems sowie ein gewisser Grad an Zuverlässigkeit.
\subsubsection{\freq{} Sicherstellung der Platformunabhängigkeit}

\section{Konzept und Modellierung}
\label{konzept_und_modellierung}
\add{SONNTAG FERTIG}
\add{Intro}

\subsection{Designentscheidungen}
\label{konzept_und_modellierung_designentscheidungen}

\subsection{...}