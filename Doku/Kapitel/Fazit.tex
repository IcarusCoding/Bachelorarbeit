\chapter{Fazit}
\label{fazit}
In dieser Arbeit wurde eine auf Java Annotationen basierende Bibliothek zur Simplifizierung und Expansion von JavaFX konzipiert und implementiert. Vorhandene Funktionen wurden in einer Problemanalyse auf Komplexität und Vollständigkeit untersucht und die daraus resultierenden Probleme dienten als Grundlage für die Anforderungsanalyse. Die identifizierten funktionalen und nicht-funktionalen Anforderungen wurden für die Entwicklung eines Konzeptes genutzt, welches wiederum das Fundament der vollständigen Systemimplementierung bildete und dabei einen besonderen Fokus auf die explizite Erfüllung aller Ziele gesetzt hat. In der prototypischen Implementierung wurden alle obligaten Anforderungen mit spezieller Berücksichtigung von paradigmatischen softwaretechnischen Qualitätsrichtlinien wie der Erweiterbarkeit und der Wartbarkeit eines Systems erfüllt. Abschließend wurde der Funktionsumfang von JavaFX mit dem von \texttt{SimpliFX} im Rahmen der Evaluation verglichen. 

\section{Zusammenfassung}
\label{zusammenfassung}

\add{Zusammenfassung}

\section{Ausblick und mögliche Erweiterungen}
\label{ausblick_und_mögliche_erweiterungen}
Im Folgenden werden mögliche Erweiterungen der implementierten Bibliothek vorgestellt. Auch wenn ein Großteil der Anforderungen durch das System erfüllt werden, existieren zu diesem Zeitpunkt noch einige optionale Anforderungen, welche nur teilweise bis gar nicht implementiert worden sind. Dazu gehört beispielsweise eine Annotationsvalidierung zur Kompilierzeit einer auf \texttt{SimpliFX} basierenden Applikation (\autoref{freq22}), was zu einer Detektion von Syntaxfehlern oder eventueller Inkorrektheiten in Konfigurationen führt. Wird eine solche Fehlkonfiguration entdeckt, kann ein Kompilierfehler durch einen Annotationsprozessor ausgelöst werden und Laufzeitausnahmen somit effektiv vermieden werden.\\
Die Unterstützung von Konfigurationsdateien ist in den Aspekten des Dateiformats und der zugelassenen Operationen stark begrenzt. Neben dem Akzeptieren von Properties- und XML-Dateien, könnten beispielsweise noch weitere bekannte Konfigurationsformate wie \ac{json} oder \ac{yaml} durch \texttt{SimpliFX} erkannt und genutzt werden. Auch ist es momentan nicht möglich, eine schreibende Operation auf Konfigurationsdateien vorzunehmen, da zwischen Ressourcen im Klassenpfad und externen Ressourcen differenziert werden müsste und es generell kompliziert ist, Dateien zu modifizieren, welche sich innerhalb eines Java Archivs befinden.
\\...\\
Das System könnte um vom Betriebssystem des Nutzers abhängige Funktionen erweitert werden. Das Fensterdesign für eine JavaFX \texttt{Stage} kann durch die Nutzung des \ac{jni} ergänzt werden. Unter Windows kann somit beispielsweise ein Unschärfeeffekt des Fensterhintergrundes realisiert werden oder eine Modifikation der Titelleiste ermöglichen. Auch könnte eine Schnittstelle entwickelt werden, welche eine direkte Interaktion mit der Taskleiste ermöglicht und so z.B. Benachrichtigungen und Statusaktualisierungen an diese übermitteln kann.