\chapter{Demo Anwendung}
\label{appendix:demo_application}
\begin{figure}[H]
	\begin{lstlisting}[caption=Demo -- Minimaler Einstiegspunkt., captionpos=b, label=lst:entry_point_demo_full, numbers=left, xleftmargin=1.7em, framexleftmargin=1.7em, nolol]
#@StageConfig(title = "Login", icons = "icon.png")
#@ApplicationEntryPoint(MainController.class)
#@GuiceInjection(MainModule.class)
public final class DemoApplication {

    public static void main(String[] args) throws Exception {
        Reflection.addOpens("java.lang.reflect", "java.base",
				JFXTextFieldSkin.class.getModule());
        SimpliFX.launch();
    }

    #@ResourceBundle("lang.Messages")
    private II18N ii18N;

    #@ConfigSource("config/connection")
    private Properties properties;

    #@Shared
    private SharedReference<StringProperty> titleRef;

    #@EventHandler
    private void onStart(StartEvent event) {
        this.titleRef.set(event.getStage().titleProperty());
        event.getStage().show();
    }

}
	\end{lstlisting}
\end{figure}
\begin{figure}[H]
	\begin{lstlisting}[caption=Demo -- \texttt{MainController}., captionpos=b, label=lst:main_controller, numbers=left, xleftmargin=1.7em, framexleftmargin=1.7em, nolol]
#@Controller(fxml = "/fxml/MainController.fxml", 
#					 css = "css/main.css")
public final class MainController {

	#@FXML
	private BorderPane root;
	
	#@Setup
	private void setup(ControllerSetupContext ctx) {
		// Erstelle Untergruppe "titleBar" mit TitleBarController 
		// als Startcontroller
	    ctx.createSubGroup(TitleBarController.class, "titleBar", 
				this.root::setTop);
		// Erstelle Untergruppe "mainContent" mit LoginController
		// als Startcontroller
	    ctx.createSubGroup(LoginController.class, "mainContent",
				this.root::setCenter);
	}

}
	\end{lstlisting}
\end{figure}
\begin{figure}[H]
	\begin{lstlisting}[caption=Demo -- \texttt{LoginController}., captionpos=b, label=lst:login_controller, numbers=left, xleftmargin=1.7em, framexleftmargin=1.7em, nolol]
#@Controller(fxml = "/fxml/LoginController.fxml")
public final class LoginController {

	#@FXML
    private JFXTextField usernameField;

    #@FXML
    private JFXPasswordField passwordField;

    #@Shared
    private SharedReference<String> usernameRef;

    #@Shared
    private SharedReference<StringProperty> titleRef;

    #@Inject
    private ILoginService loginService;

    private ControllerGroupContext ctx;

    #@Setup
    private void setup(ControllerSetupContext ctx) {
        ctx.preloadController(MainMenuController.class);
        this.ctx = ctx.getGroupContext();
    }

    #@OnShow
    private void onShow() {
        this.titleRef.get().set("Login");
    }

    #@OnHide
    private void onHide() {
        this.usernameField.clear();
        this.passwordField.clear();
    }

    #@FXML
    private void onLogin() {
        if (loginService.login(this.usernameField.getText(),
				this.passwordField.getText())) {
            this.usernameRef.set(this.usernameField.getText());
            this.ctx.switchController(MainMenuController.class,
					new FadeAnimation(Duration.millis(250)));
        }
    }

}
	\end{lstlisting}
\end{figure}
\begin{figure}[H]
	\begin{lstlisting}[caption=Demo -- \texttt{MainMenuController}., captionpos=b, label=lst:mainmenu_controller, numbers=left, xleftmargin=1.7em, framexleftmargin=1.7em, nolol]
#@Controller(fxml = "/fxml/MainMenuController.fxml")
public final class MainMenuController {

    #@FXML
    private BorderPane root;

    #@FXML
    private StackPane contentCenter;

    #@ResourceBundle
    private II18N ii18N;

    #@Shared
    private SharedResources resources;

    private StringBinding binding;

    #@Setup
    private void setup(ControllerSetupContext ctx) {
        ctx.createSubGroup(SidebarController.class, "sidebar",
				this.root::setLeft);
        ctx.createSubGroup(TestControllerOne.class,
				"sidebarContent",
				this.contentCenter.getChildren()::setAll);
    }

    #@PostConstruct
    private void afterConstruction() {
        this.binding = this.ii18N
				.createObservedBinding("mainMenu.welcome",
						this.resources.getForName("username")
						.asProperty());
    }

    #@OnShow
    private void onShow() {
        final SharedReference<StringProperty> prop = 
				this.resources.getForName("title");
        prop.get().set(this.binding.get());
    }

}
	\end{lstlisting}
\end{figure}